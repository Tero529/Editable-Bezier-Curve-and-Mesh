\hypertarget{index_Contributors}{}\section{Contributors}\label{index_Contributors}

\begin{DoxyItemize}
\item Varun Natu 2014\+A7\+P\+S841H
\item Ayush Sharma 2014\+A7\+P\+S039H
\item Akanksha Pandey 2014\+A7\+P\+S151H
\end{DoxyItemize}\hypertarget{index_Implementation}{}\section{Implementation}\label{index_Implementation}
\hypertarget{index_algo1}{}\subsection{Editable Bezier Curve}\label{index_algo1}
A blank black canvas is created and displayed to the screen which is used to build bezier curves.
\begin{DoxyItemize}
\item Right clicking anywhere on the screen adds a control point and shows its effect on the bezier curve in real time.
\item A left click in a 5 pixel vicinity of a control point leads to it\textquotesingle{}s removal from the control point list, which is reflected in real time on the curve
\item To be able to drag a point, the d key has to be pressed first to enable dragging mode after which any control point can be dragged while the right mouse key is pressed
\item A convex hull of the control points is displayed as red dotted lines. The list of control points is converted to a bezier curve using the De Castlejau recursive algorithm \begin{DoxySeeAlso}{See also}
\hyperlink{main_8cpp}{main.\+cpp} \hyperlink{bezier_8hpp}{bezier.\+hpp} \hyperlink{task2_8hpp}{task2.\+hpp}
\end{DoxySeeAlso}

\end{DoxyItemize}\hypertarget{index_algo2}{}\subsection{Surface of Revolution and Mesh Generation}\label{index_algo2}
Once the user has finished editing his bezier curve on the given canvas, and clicks the escape key to exit the application, the curve is sampled at a t interval value of 0.\+1 and a surface of revolution is created by rotation about the x axis at a sampled angle of ten degrees. This surface of revolution is stored in a Triangular Vertex-\/\+Face mesh data structure and written to an ouput O\+FF file. \begin{DoxySeeAlso}{See also}
\hyperlink{main_8cpp}{main.\+cpp} \hyperlink{task34_8hpp}{task34.\+hpp}
\end{DoxySeeAlso}
\hypertarget{index_extra_algos}{}\subsection{Example Images}\label{index_extra_algos}
12 examples, 3 each for degrees 2,3,4,5 as a set of Curve,Mesh and O\+FF files can be found in the mentioned H\+T\+ML page. 